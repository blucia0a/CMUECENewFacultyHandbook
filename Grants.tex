\section{Funding and Grant Proposals}

As a new faculty member, you will need to apply for grants to fund your
research group.  The process of applying for a grant is divided into three
parts.  The first part is what you need to do before you can submit the grant
proposal.  The second part is what you need to do to actually submit the grant
proposal.  The third part is what you need to do after a proposal has been
funded.  The department and university employ people that help support faculty
applying for grants during each of these steps.  The purpose of the next three
sections is to give you an idea of what you have to do and to make you aware of
what parts of the process are institutionally supported.  This section is
written with an emphasis on the process for submitting grant proposals to the
NSF, and the process may vary slightly for other funding organizations.

\subsection{Pre-submission Tasks}
Before you submit a grant proposal, you need to identify a target call for
proposals.  The call that you submit to will be different depending on your
area of research and the best way to know what to apply to is to ask someone
with experience submitting proposals in your area.  

\subsubsection{Selecting an NSF program}
The NSF has many avenues to funding.  At the top level, you will target a
particular {\em directorate}.  A directorate is made up of {\em organizations}.
Within each of a directorate's organizations is a collection of {\em programs}.
Programs are built around the specific solicitations that you will respond to
with your proposal.  

A special category of NSF programs that are part of some organizations are the
{\em core programs}, which broadly call for proposals relevant to an
organization each year.  You are limited to two NSF core programs proposals per
year, so be strategic with what you submit.  

Here is an illustrative example of the structure of NSF. If you are a computer
systems researcher, you might submit a proposal to the CISE (Computer and
Information Science and Engineering) directorate.  Within CISE, you must then
select whether your work best fits the Advanced Cyberinfrastructure (AC),
Computing and Communication Foundations (CCF), Computer and Network Systems
(CNS), or Information \& Intelligent Systems (IIS) organizations.  Having
selected an organization --- CCF, for instance --- you may then decide to
submit to the Core Programs, focusing on one of Algorithmic Foundations (AF),
Communications and Information Foundations (CIF), or Software and Hardware
Foundations (SHF).  A hypothetical submission may then be categorized as CISE:
CCF: SHF.  Note that when you submit your proposal, the call for proposals asks
you to put that long prefix in the name of your proposal so its categorization
is obvious.  You should read through the different available directorates,
organizations, and programs to find the one(s) that are the best match for your
work.  If you are unsure, ask others in your area where they think your work
belongs.  People that have done this before can probably help you aim your
proposal at the right target.

{\noindent \em \bf Program Managers.}
A directorate's organizations' programs are each managed by someone called a
{\em program manager}.  The program manager collects applications, tweaks the
call for proposals to focus on topics that are currently important, and
solicits experts (i.e., your academic peers) to serve on proposal review
panels.  Getting to know the program manager is a good idea, because you can
let him or her know about your work and its importance.  It is also a good idea
to reach out to program managers to avail yourself to serve on proposal review
panels.  

\subsubsection{Preparing Your Submission}
After selecting a program, you need to start preparing your submission.  The
details of how to prepare a good grant proposal are not in this document.
However, there are many great references for preparing a good grant proposal,
including the spiral bound grant writing handbook that Jelena has several
copies of in her office.

The first step to submitting a grant proposal is to alert the departmental
sponsored research administration team that you are planning to submit.  {\em
You need to do this well in advance of the submission deadline.}  The sponsored
research administration team reads the fine print on the proposal that you are
submitting to, works with you to prepare a budget, and can help to get
documents that you produce organized and submitted to the right place at the
right time.  When you start your job, you should be sure to figure out who your
contact person is in the sponsored research administration team. 

{\noindent \em \bf The Budget.} NSF programs, as well as most other formal
calls for grant proposals require you to submit a meticulously prepared budget
listing how much money you are asking for and accounting for how you plan to
use it.  The sponsored research administration team has seen many budgets come
and go and they are skilled at using the right language to describe the budget.
A typical process will be that you will read the call for proposals, then you
will give an overview of the amount of money you are asking for, as well as
your resource needs to your contact in the sponsored research administration
team.  Your contact will then get back to you with a budget that matches your
constraints, if possible.  If the amount you asked for is somehow weird, or you
cannot afford to fund all of the resources that you asked to be included in the
budget, your admin will work with you to converge on a budget that works. 

{\noindent \em \bf NSF Fastlane}  

You submit your grant proposal to the NSF Fastlane website
(https://www.fastlane.nsf.gov/fastlane.jsp).   You should talk to your grant
administrator about getting an NSF ID.  There is an internal system and a
system at NSF that the sponsored research people need to make sure are in sync.
Setting up this ID takes some time, so ask the sponsored research
administration people to help you with that early on.

The Fastlane website is a little shady.  It works, but it feels clunky and
uncertain a lot of the time.  Always double check that your documents are all
what you think they are.  Once you're at the Fastlane website, you'll want to
go to ``Proposals, Awards and Status'' and log in.  After that, follow the
menus to add pieces to your proposal.  You are probably going to want the
sponsored research office to have access to your account (you can work that out
with them).  That way, they can fill in boilerplate documents based on
information you give them and you can proxy your submission documents to them
if you want to.  


\subsection{Submission Tasks}

The proposal consists of a lot more than just the documents describing your
research plan, broader impact, and significance (although that is the hardest
part).  You should make sure that you save time to prepare your BioSketch, your
Data Management Plan, and any other supplementary documents (like letters of
collaboration).  The first time you submit something, these might take some
time to prepare.  After you do them once, they are easy to ``port'' from one
submission to the next.  The precise structure and content of these documents
might vary by field, so the best thing to do is to ask someone that you know to
let you borrow one that they have submitted before.  Doing that will give you
an idea of what works.

\subsection{Post-funding Tasks}

When you get a proposal funded there are a few more things you need to do
before you have access to your funding.  First, you should go get a drink or
some ice cream or something, to celebrate;  Nice job!  

Second, you will have to interact with your program manager (who you may have
been interacting with already).  Your program manager will tell you what the
next steps are.  At a minimum, you will have to submit The Official Abstract of
Public Record for your proposal.  A program manager for a past proposal that
was accepted said this of the abstract of public record:

  "The abstract is what NSF publishes as the public record of the research.
  These abstracts are mined very very very heavily by Congress, the media, and
  others.  It is really important that the first paragraph or two of the abstract
  explain the work so that a broad non-science audience can both understand and
  see the value of the investment in this area.  The final paragraph or two may
  contain the technical details allowing the computer science community to
  appreciate the project. 

  The abstract should not be more than 1 page and 3/4 of a page is actually better.
  Things that end up both good and bad in the NY Times, in the Colburn reports, etc, 
  are taken from the abstract.  So, both Title and Abstract need to be crystal clear, 
  well-explained, and documentary of the importance of the research.  "

Writing this abstract is a victory lap, but according to this program manager,
it is a very important victory lap.

Third, you need to contact the sponsored research office again, to tell them
that you have some incoming funding.  They work with the ECE business office to
create an ``account string''.  The account string is something that identifies
your money.  Your admin can usually handle the details of billing things to
your various account strings.   For instance, if you need to buy some stuff or
fund a student, and you've received funding from two different NSF proposals,
you will have to choose which proposal (and account string) to draw from.   

\subsection{Overhead: Grants, Awards, and Gifts}

The university takes most of your money from you after you win it.  Thems the
breaks.  This practice is known as ``overhead'' or as ``indirect cost''. The
rate of indirect cost varies, but it is currently around 55\% for government
agency grant proposals.   

If you talk to a company and they agree to give you an ``unrestricted gift'',
overhead works differently.  Companies usually do not allow a university to
take overhead on gift funds and that provision is in the call for proposals.
To get around that provision, the sponsored research office has begun charging
an ``adminstrative fee'' of 12\% on gift funds.

For government agency grant proposals, it is usually obvious that you are
receiving grant funding subject to the 55\% overhead.  With industry funding it
is sometimes not totally clear and can vary on a case-by-case basis.  When you
reach a funding agreement with a company, it is often good to ask them directly
whether they are granting you money as an ``unrestricted gift'', or whether the
funding is part of a ``contract'' or ``research agreement'', both of which may
fall under the 55\% overhead rule, or some other obscure rule governing your
particular agreement.  If you have any doubt, talk to the sponsored research
office.
