\section{Funding and Grant Proposals}

As a new faculty member, you will need to apply for grants to fund your
research group.  The process of applying for a grant is divided into three
parts.  The first part is what you need to do before you can submit the grant
proposal.  The second part is what you need to do to actually submit the grant
proposal.  The third part is what you need to do after a proposal has been
funded.  The department and university employ people that help support faculty
applying for grants during each of these steps.  The purpose of the next three
sections is to give you an idea of what you have to do and to make you aware of
what parts of the process are institutionally supported.  This section is
written with an emphasis on the process for submitting grant proposals to the
NSF, and the process may vary slightly for other funding organizations.

\subsection{Pre-submission Tasks}
Before you submit a grant proposal, you need to identify a target call for
proposals.  The call that you submit to will be different depending on your
area of research and the best way to know what to apply to is to ask someone
with experience submitting proposals in your area.  

\subsubsection{Selecting an NSF program}
The NSF has many avenues to funding.  At the top level, you will target a
particular {\em directorate}.  A directorate is made up of {\em organizations}.
Within each of a directorate's organizations is a collection of {\em programs}.
Programs are built around the specific solicitations that you will respond to
with your proposal.  

A special category of NSF programs that are part of some organizations are the
{\em core programs}, which broadly call for proposals relevant to an
organization each year.  You are limited to two NSF core programs proposals per
year, so be strategic with what you submit.  

Here is an illustrative example of the structure of NSF. If you are a computer
systems researcher, you might submit a proposal to the CISE (Computer and
Information Science and Engineering) directorate.  Within CISE, you must then
select whether your work best fits the Advanced Cyberinfrastructure (AC),
Computing and Communication Foundations (CCF), Computer and Network Systems
(CNS), or Information \& Intelligent Systems (IIS) organizations.  Having
selected an organization --- CCF, for instance --- you may then decide to
submit to the Core Programs, focusing on one of Algorithmic Foundations (AF),
Communications and Information Foundations (CIF), or Software and Hardware
Foundations (SHF).  A hypothetical submission may then be categorized as CISE:
CCF: SHF.  Note that when you submit your proposal, the call for proposals asks
you to put that long prefix in the name of your proposal so its categorization
is obvious.

{\noindent \em \bf Program Managers.}
A directorate's organizations' programs are each managed by someone called a
{\em program manager}.  The program manager collects applications, tweaks the
call for proposals to focus on topics that are currently important, and
solicits experts (i.e., your academic peers) to serve on proposal review
panels.  Getting to know the program manager is a good idea, because you can
let him or her know about your work and its importance.  It is also a good idea
to reach out to program managers to avail yourself to serve on proposal review
panels.  

\subsubsection{Preparing Your Submission}
After selecting a program, you need to start preparing your submission.  The
details of how to prepare a good grant proposal are not in this document.
However, there are many great references for preparing a good grant proposal,
including \xxx{TODO: that book that Jelena gave Brandon}.

The first step to submitting a grant proposal is to alert the departmental
sponsored research administration team that you are planning to submit.  {\em
You need to do this well in advance of the submission deadline.}  The sponsored
research administration team reads the fine print on the proposal that you are
submitting to, works with you to prepare a budget, and can help to get
documents that you produce organized and submitted to the right place at the
right time.  When you start your job, you should be sure to figure out who your
contact person is in the sponsored research administration team. 

{\noindent \em \bf The Budget.} NSF programs, as well as most other formal
calls for grant proposals require you to submit a meticulously prepared budget
listing how much money you are asking for and accounting for how you plan to
use it.  The sponsored research administration team has seen many budgets come
and go and they are skilled at using the right language to describe the budget.
A typical process will be that you will read the call for proposals, then you
will give an overview of the amount of money you are asking for, as well as
your resource needs to your contact in the sponsored research administration
team.  Your contact will then get back to you with a budget that matches your
constraints, if possible.  If the amount you asked for is somehow weird, or you
cannot afford to fund all of the resources that you asked to be included in the
budget, your contact will work with you to converge on a budget that works. 

{\noindent \em \bf Fastlane and Grants.gov}  \xxx{TODO: Talk about shady website quirks,
submitting early, etc}.


\subsection{Submission Tasks}

\xxx{talk about "last mile" things, like DMP, bio, etc

\subsection{Post-funding Tasks}

\xxx{Someone besides Brandon should write this.}

\subsection{Overhead: Grants, Awards, and Gifts}

\xxx{54\% on grants, 12\% admin fee on awards and gifts.  When in doubt, contact someone that knows what counts as what.}
